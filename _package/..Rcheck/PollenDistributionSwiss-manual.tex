\nonstopmode{}
\documentclass[letterpaper]{book}
\usepackage[times,inconsolata,hyper]{Rd}
\usepackage{makeidx}
\makeatletter\@ifl@t@r\fmtversion{2018/04/01}{}{\usepackage[utf8]{inputenc}}\makeatother
% \usepackage{graphicx} % @USE GRAPHICX@
\makeindex{}
\begin{document}
\chapter*{}
\begin{center}
{\textbf{\huge Package `PollenDistributionSwiss'}}
\par\bigskip{\large \today}
\end{center}
\ifthenelse{\boolean{Rd@use@hyper}}{\hypersetup{pdftitle = {PollenDistributionSwiss: Analyzing and Visualizing Pollen Distribution in Switzerland}}}{}
\ifthenelse{\boolean{Rd@use@hyper}}{\hypersetup{pdfauthor = {William Ambrosetti; Alessia Bernacchia; Alessio Carnevale; Fernandez Sergio; Federico Gritti}}}{}
\begin{description}
\raggedright{}
\item[Type]\AsIs{Package}
\item[Title]\AsIs{Analyzing and Visualizing Pollen Distribution in Switzerland}
\item[Version]\AsIs{0.1.0}
\item[Author]\AsIs{See Authors@R}
\item[Maintainer]\AsIs{William Ambrosetti }\email{william.ambrosetti@student.supsi.ch}\AsIs{}
\item[Depends]\AsIs{R (>= 4.1.0)}
\item[Description]\AsIs{More about what it does (maybe more than one line).
Continuation lines should be indented.}
\item[License]\AsIs{GPL (>= 3)}
\item[Encoding]\AsIs{UTF-8}
\item[Imports]\AsIs{ggplot2, dplyr, sf, tidyr, httr, jsonlite, httr2, testthat,
knitr, readr, tibble, magrittr, terra, utils}
\item[RoxygenNote]\AsIs{7.3.2}
\end{description}
\Rdcontents{Contents}
\HeaderA{compare\_pollen\_levels}{Compare pollen levels between two cities}{compare.Rul.pollen.Rul.levels}
%
\begin{Description}
Compare pollen levels between two cities
\end{Description}
%
\begin{Usage}
\begin{verbatim}
compare_pollen_levels(
  city1_name,
  city2_name,
  city1_lat,
  city1_lon,
  city2_lat,
  city2_lon
)
\end{verbatim}
\end{Usage}
%
\begin{Arguments}
\begin{ldescription}
\item[\code{city1\_name}] Name of the first city

\item[\code{city2\_name}] Name of the second city

\item[\code{city1\_lat}] Latitude of the first city

\item[\code{city1\_lon}] Longitude of the first city

\item[\code{city2\_lat}] Latitude of the second city

\item[\code{city2\_lon}] Longitude of the second city
\end{ldescription}
\end{Arguments}
%
\begin{Value}
A string with the name of the city that has the least pollen
\end{Value}
%
\begin{Examples}
\begin{ExampleCode}
city1_name <- "Zurich"
city1_lat <- 47.3769
city1_lon <- 8.5417

city2_name <- "Geneva"
city2_lat <- 46.2044
city2_lon <- 6.1432
result <- compare_pollen_levels(city1_name, city2_name, city1_lat, city1_lon, city2_lat, city2_lon)

\end{ExampleCode}
\end{Examples}
\HeaderA{get\_pollen\_forecast}{Getting the pollen data from the API}{get.Rul.pollen.Rul.forecast}
%
\begin{Description}
Load the forecast data about pollution of today, in the given latitude and longitude.
\end{Description}
%
\begin{Usage}
\begin{verbatim}
get_pollen_forecast(latitude, longitude)
\end{verbatim}
\end{Usage}
%
\begin{Arguments}
\begin{ldescription}
\item[\code{latitude}] Latitude of the location

\item[\code{longitude}] Longitude of the location
\end{ldescription}
\end{Arguments}
%
\begin{Value}
A data.frame with pollen forecast data
\end{Value}
%
\begin{Examples}
\begin{ExampleCode}
latitude <- 47.3769
longitude <- 8.5417

df_test <- get_pollen_forecast(latitude, longitude)

\end{ExampleCode}
\end{Examples}
\HeaderA{get\_pollen\_forecast\_with\_cache}{Read data from the API, uses caching}{get.Rul.pollen.Rul.forecast.Rul.with.Rul.cache}
%
\begin{Description}
Checks if a file with the requested data exists, if so it uses that file to return the requested data,
otherwise it makes an API call.
\end{Description}
%
\begin{Usage}
\begin{verbatim}
get_pollen_forecast_with_cache(
  latitude,
  longitude,
  PATH = "pollen_cache/pollen_%.4f_%.4f_%s.rds"
)
\end{verbatim}
\end{Usage}
%
\begin{Arguments}
\begin{ldescription}
\item[\code{latitude}] Latitude of the location

\item[\code{longitude}] Longitude of the location

\item[\code{PATH}] path to the cache
\end{ldescription}
\end{Arguments}
%
\begin{Value}
A data.frame with pollen forecast data
\end{Value}
%
\begin{Examples}
\begin{ExampleCode}
latitude <- 47.3769
longitude <- 8.5417

df_test <- get_pollen_forecast_with_cache(latitude, longitude)

\end{ExampleCode}
\end{Examples}
\HeaderA{get\_pollen\_for\_list\_cities}{Get data pollen of a given city}{get.Rul.pollen.Rul.for.Rul.list.Rul.cities}
%
\begin{Description}
Retrieve information about the pollen of given cities
\end{Description}
%
\begin{Usage}
\begin{verbatim}
get_pollen_for_list_cities(list_cities, PATH = "SwissCities.csv")
\end{verbatim}
\end{Usage}
%
\begin{Arguments}
\begin{ldescription}
\item[\code{list\_cities}] list of cities in a string form

\item[\code{PATH}] path of the cities file
\end{ldescription}
\end{Arguments}
%
\begin{Value}
A tibble with city name, canton, lat, lng, pollin level
\end{Value}
%
\begin{Examples}
\begin{ExampleCode}
## Not run: 
tibble_test <- get_pollen_for_list_cities(c("Lugano", "Geneva"), "SwissCities.csv")

## End(Not run)

\end{ExampleCode}
\end{Examples}
\HeaderA{plot\_most\_canton}{Plot the Pollins in the Swiss Map}{plot.Rul.most.Rul.canton}
%
\begin{Description}
Plots the distribution of the pollines across the cities given
\end{Description}
%
\begin{Usage}
\begin{verbatim}
plot_most_canton(polline_df, PATH = "2025_GEOM_TK")
\end{verbatim}
\end{Usage}
%
\begin{Arguments}
\begin{ldescription}
\item[\code{polline\_df}] pollin df

\item[\code{PATH}] path for the folder
\end{ldescription}
\end{Arguments}
%
\begin{Value}
plot of most prominent pollin on the swiss map
\end{Value}
%
\begin{Examples}
\begin{ExampleCode}
## Not run: 
city_df <- get_pollen_for_list_cities(c("Geneva", "Lugano"))
plot_most_canton(city_df)
## End(Not run)

\end{ExampleCode}
\end{Examples}
\HeaderA{plot\_specific\_pollen}{Calculation of average pollen by canton}{plot.Rul.specific.Rul.pollen}
%
\begin{Description}
This function takes a data frame with pollen concentration data and generates
a thematic map of Switzerland, showing the average concentration of a specific
pollen type (tree, weed, or grass) across cantons. It overlays the pollen data
on top of canton shapefiles and lakes for context.
\end{Description}
%
\begin{Usage}
\begin{verbatim}
plot_specific_pollen(
  polline_df,
  pollen_type = "grass",
  PATH_map = "2025_GEOM_TK"
)
\end{verbatim}
\end{Usage}
%
\begin{Arguments}
\begin{ldescription}
\item[\code{polline\_df}] A data frame containing pollen concentration data, including canton names
and numeric columns for each pollen type (e.g., grass\_pollin, tree\_pollin, etc.).

\item[\code{pollen\_type}] Character string indicating the type of pollen to visualize.
Must correspond to a column prefix in polline\_df (e.g., "grass", "tree", or "weed").

\item[\code{PATH\_map}] path for map directory
\end{ldescription}
\end{Arguments}
%
\begin{Value}
plot of most prominent pollin on the swiss map
\end{Value}
%
\begin{Examples}
\begin{ExampleCode}
## Not run: 
df <- data.frame(canton = c("Zürich", "Bern"),
                 grass_pollin = c(30, 50),
                 tree_pollin = c(10, 20),
                 weed_pollin = c(5, 15))

plot_specific_pollen(df, pollen_type = "grass", PATH='2025_GEOM_TK')

## End(Not run)

\end{ExampleCode}
\end{Examples}
\printindex{}
\end{document}
